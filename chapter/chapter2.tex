\chapter{线性方程组}

\section{向量的线性相关性}

\subsection{向量的定义}

\begin{definition}
    \label{def-vector}
    由$n$个数$a_1,\cdots,a_n$顺序构成的$n$元有序数组称为$n$元向量。记作
    \begin{equation}
        \alpha=(a_1,a_2,\cdots,a_n),\label{eq-vertor-1}
    \end{equation}
    $a_i$称为向量$\alpha$的第$a_i,i=1,2,\cdots,n$个分量。(\ref{eq-vertor-1})的形式称为行向量,除了行向量形式之外,向量还可以写成列向量的形式
    \begin{eqnarray}
        \alpha=\begin{bmatrix}
            a_1\\\vdots\\a_n
        \end{bmatrix}=(a_1,\cdots,a_n)^T.
    \end{eqnarray}
\end{definition}

向量的相等、加法、减法、数乘等操作参照矩阵的对应运算,不做赘述。

\subsection{向量的线性相关性}

\begin{definition}
    \label{def-linear-composition}
    设$\alpha_1,\cdots,\alpha_m$是$m$个$n$元向量,$k_1,\cdots,k_m$是任意$m$个数,称下列向量
    \[ \beta=k_1\alpha_1+\cdots+k_m\alpha_m \]
    是向量组的$\alpha_1,\cdots,\alpha_m$的一个线性组合。此时,也称向量$\beta$可由向量组$\alpha_1,\cdots,\alpha_m$线性表出。
\end{definition}

\begin{definition}
    \label{def-linear-relation}
    设$\alpha_1,\cdots,\alpha_m$是$m$个$n$元向量,若存在$m$个不全为零的数$k_1,\cdots,k_m$使得
    \begin{equation}
        k_1\alpha_1+\cdots+k_m\alpha_m=0\label{eq-linear-dependence}
    \end{equation}
    则称向量组$\alpha_1,\cdots,\alpha_m$线性相关,否则称为线性无关。
\end{definition}

\begin{remark}
    平行(共线)是描述两个向量之间关系的概念,而共面是描述多个向量之间关系的概念,而线性相关(线性无关)则是一个更高层次的概念。
\end{remark}

\begin{remark}
    线性相关性是从共线、共面等概念引申出来的应用于更高维向量的概念。在二维空间中,向量$\gamma,\eta$的共线即表示存在两个不全为零的实数$k_1,k_2$使得$k_1\gamma+k_2\eta=0$(这里隐含地说明了为什么零向量共线于任何向量);在三维空间中,向量$\gamma,\eta,\varepsilon$共面即表示存在三个不全为零的实数$k_1,k_2,k_3$使得$k_1\gamma+k_2\eta+k_3\varepsilon=0$。
\end{remark}

基于线性相关和线性无关的定义,可以到处一些基本结论(证明从略)。

\begin{thm}
    \label{thm-iff-vector-group-dependence}
    向量组$\alpha_1,\cdots,\alpha_m$线性相关当且仅当至少存在$\alpha_i,1\leqslant i\leqslant m$使可以由其余的向量线性表出。
\end{thm}

\begin{thm}
    \label{thm-independence-vec-grp-append}
    向量组$\alpha_1,\cdots,\alpha_m$线性无关,向量组$\beta,\alpha_1,\cdots,\alpha_m$,则$\beta$可以由$\alpha_1,\cdots,\alpha_m$线性表出,且表示法唯一。
\end{thm}

\begin{definition}
    \label{def-linear-represent}
    设$\alpha_1,\cdots,\alpha_s$与$\beta_1,\cdots,\beta_t$是两组$n$元向量,若每个$\alpha_i,i=1,2,\cdots,s$均可以由$\beta_1,\cdots,\beta_t$线性表出,则称向量组$\alpha_1,\cdots,\alpha_s$可以由向量组$\beta_1,\cdots,\beta_t$线性表出。若向量组$\alpha_1,\cdots,\alpha_s$与$\beta_1,\cdots,\beta_t$可以相互线性表出,则称$\alpha_1,\cdots,\alpha_s$与$\beta_1,\cdots,\beta_t$等价,记作
    \[ \{\alpha_1,\cdots,\alpha_s\}\cong\{\beta_1,\cdots,\beta_t\} \]
\end{definition}

向量组的等价是一种等价关系,拥有自反性、对称性和传递性。

关于向量组线性相关性,有一个重要的充分定理。

\begin{thm}
    \label{thm-important-linear-dependence}
    设$\alpha_1,\cdots,\alpha_s$是一组$n$元向量。若存在另一组$n$元向量$\beta_1,\cdots,\beta_t$有
    \begin{enumerate}
        \item $\alpha_1,\cdots,\alpha_s$可以由向量组$\beta_1,\cdots,\beta_t$线性表出;
        \item $s>t$,
    \end{enumerate}
    则向量组$\alpha_1,\cdots,\alpha_s$线性相关。
\end{thm}

定理的逆否推论如下

\begin{corollary}
    \label{col-important-linear-independence}
    设$\alpha_1,\cdots,\alpha_s$与$\beta_1,\cdots,\beta_t$是两组$n$元向量,若存在
    \begin{enumerate}
        \item $\alpha_1,\cdots,\alpha_s$可以由向量组$\beta_1,\cdots,\beta_t$线性表出;
        \item $\alpha_1,\cdots,\alpha_s$线性无关,
    \end{enumerate}
    则$s\leqslant t$。
\end{corollary}

定理\ref{thm-important-linear-dependence}的证明如下,由于定理\ref{col-important-linear-independence}是前者的逆否,只要证明了前者,则后者正确性显然。证明如下。

\begin{proof}
    \label{proof-important-linear-dependence}

    由条件1可知,存在$c_{ij},i=1,\cdots,s;j=1,\cdots,t$使得
    \[ \alpha_i=c_{i1}\beta_1+\cdots+c_{it}\beta_t \]
    设
    \[ \gamma_i=(c_{i1},\cdots,c_{it}),i=1,\cdots,s \]
    则$\gamma_1,\cdots,\gamma_s$为$s$个$t$元向量。

    由于$s>t$,根据引理\ref{lenma-vector-linear-dependence},$\gamma_1,\cdots,\gamma_s$线性相关,存在不全为零的数$k_1,\cdots,k_s$使得
    \[ k_1\gamma_1+\cdots+k_s\gamma_s=0 \]
    即
    \[ k_1c_{1j}+\cdots+k_sc_{sj}=\sum_{i=1}^sk_ic_{ij}=0,j=1,\cdots,s \]
    又因为
    \begin{equation*}
        \begin{aligned}
            \sum_{i=1}^sk_i\alpha_i
            =& \sum_{i=1}^sk_i\left(\sum_{j=1}^tc_{ij}\beta_j\right) \\
            =& \sum_{i=1}^s\sum_{j=1}^tk_ic_{ij}\beta_j \\
            =& \sum_{j=1}^t\sum_{i=1}^sk_ic_{ij}\beta_j \\
            =& \sum_{j=1}^t0\beta_j=0
        \end{aligned}
    \end{equation*}
    且$k_1,\cdots,k_t$不全为零,所以$\alpha_1,\cdots,\alpha_s$线性相关。
\end{proof}

\begin{lemma}
    \label{lenma-vector-linear-dependence}
    $m$个$n$元向量($m>n$)线性相关。
\end{lemma}

引理\ref{lenma-vector-linear-dependence}的证明就是将向量线性组合展开成为方程组,观察到方程组未知数个数大于方程个数,直接得出方程组存在非零解,从而认定,向量线性相关。

\section{向量组的秩}

\subsection{向量组的秩}

对于线性相关的向量组,秩可以描述其线性相关性的强弱程度。由此我们提出秩的概念。

\begin{definition}
    \label{def-vector-group-rank}
    设向量组$\alpha_1,\cdots,\alpha_m$有$m$个$n$元向量。其中存在$r$个向量线性无关,但是任意的$r+1$个向量都线性相关,则称向量组$\alpha_1,\cdots,\alpha_m$的值为$r$,记作$r\{\alpha_1,\cdots,\alpha_m\}=r$。
\end{definition}

\begin{remark}
    向量组的秩为$r$并不意味着向量组中任取$r$个向量都线性无关,例如向量组
    \[
        a=(1,1,0,0),b=(1,1,0,0),c=(0,1,0,0),d=(0,1,0,0)
    \]
    其秩为$2$,但是若取部分组$a,b$,显然不是线性无关的。
\end{remark}

向量组的秩可以来衡量向量组的线性相关性,通过其定义,容易给出如下充要条件。

\begin{thm}
    \label{thm-vector-group-dependence-rank}
    向量组$\alpha_1,\cdots,\alpha_m$线性相关当且仅当$r\{\alpha_1,\cdots,\alpha_m\}<m$。
\end{thm}

为了确定向量组的秩,我们引入一些相关的概念。

\begin{definition}
    \label{def-max-independence-group}
    设向量组$\alpha_1,\cdots,\alpha_m$的秩为$r$,则$\alpha_1,\cdots,\alpha_m$中任意$r$个线性无关的向量都称为其极大线性无关部分组,简称极大无关组。
\end{definition}

定义了极大无关组之后,我们可以证明向量组的所有极大无关组都是等价的。

\begin{property}
    \label{property-equvilent-mig}
    向量组与其任意一极大无关组等价。
\end{property}

\begin{proof}
    设向量组$\alpha_1,\cdots,\alpha_m$的极大无关组为$\alpha_{i_1},\cdots,\alpha_{i_r}$,显然后者可以被前者线性表出,只需证明前者可以被后者线性表出即可。

    对于向量组$\alpha_1,\cdots,\alpha_m$中的任意向量$\alpha_k$,若$\alpha_k\in\{\alpha_{i_1},\cdots,\alpha_{i_r}\}$,那么可以被其线性标出。否则,向量组$\{\alpha_k,\alpha_{i_1},\cdots,\alpha_{i_r}\}$线性相关,根据定理\ref{thm-independence-vec-grp-append},$\alpha_k$被向量组$\alpha_{i_1},\cdots,\alpha_{i_r}$唯一地线性标出。所以$\alpha_1,\cdots,\alpha_m$可以被$\alpha_{i_1},\cdots,\alpha_{i_r}$线性表出。

    故向量组与其任意一极大无关组等价。
\end{proof}

更进一步地,我们有如下定理。

\begin{thm}
    \label{thm-rank-decrease-by-linear-composition}
    若向量组$\alpha_1,\cdots,\alpha_s$可以由$\beta_1,\cdots,\beta_t$线性表出,则
    \[
        r\{\alpha_1,\cdots,\alpha_s\}\leqslant r\{\beta_1,\cdots,\beta_t\}
    \]
\end{thm}

\begin{proof}
    设向量组$\alpha_1,\cdots,\alpha_s;\beta_1,\cdots,\beta_t$的极大无关组分别为$\alpha_{i_1},\cdots,\alpha_{i_r}$和$\beta_{i_1},\cdots,\beta_{i_p}$。根据定理\ref{property-equvilent-mig},从而有
    \[
        \begin{aligned}
            \{\alpha_{i_1},\cdots,\alpha_{i_r}\}\cong&\{\alpha_1,\cdots,\alpha_s\} \\
            \{\beta_1,\cdots,\beta_t\}\cong&\{\beta_{i_1},\cdots,\beta_{i_p}\}
        \end{aligned}
    \]

    而$\beta_1,\cdots,\beta_t$又可以线性表出$\alpha_1,\cdots,\alpha_s$,所以$\beta_{i_1},\cdots,\beta_{i_p}$线性表出$\alpha_{i_1},\cdots,\alpha_{i_r}$,根据推论\ref{col-important-linear-independence}有$r\leqslant p$,从而有
    \[
        r\{\alpha_1,\cdots,\alpha_s\}\leqslant\{\beta_1,\cdots,\beta_t\}
    \]
\end{proof}

极大无关组拥有如下的等价定义。由此极大无关组的定义可以参见定义\ref{def-max-independence-group}和定义\ref{def-max-independence-group-2}。

\begin{definition}
    \label{def-max-independence-group-2}
    设$\alpha_{i_1},\cdots,\alpha_{i_r}$是向量组$\alpha_1,\cdots,\alpha_m$的一个部分组,若$\alpha_{i_1},\cdots,\alpha_{i_r}$线性无关且每个$\alpha_j(j=1,2,\cdots,m)$均可以由$\alpha_{i_1},\cdots,\alpha_{i_r}$线性表出,则$\alpha_{i_1},\cdots,\alpha_{i_r}$是向量组$\alpha_1,\cdots,\alpha_m$的极大无关组。
\end{definition}

等价性证明容易给出。只需要证明对于一个向量组,根据定义\ref{def-max-independence-group}和定义\ref{def-max-independence-group-2}定义出向量组总是相等的即可。

\begin{proof}
    设向量组$\alpha_1,\cdots,\alpha_m$,根据定义\ref{def-max-independence-group}和定义\ref{def-max-independence-group-2}定义出的极大无关组分别是$V_1,V_2$,根据定义\ref{def-max-independence-group-2},显然有$V_2\subseteq V_1$。反之,根据定义\ref{def-max-independence-group}和性质\ref{property-equvilent-mig},也能有$V_1\subseteq V_2$。所以有$V_1=V_2$,故定义\ref{def-max-independence-group}和定义\ref{def-max-independence-group-2}等价。
\end{proof}