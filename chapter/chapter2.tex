\chapter{线性方程组}

\section{向量的线性相关性}

\subsection{向量的定义}

\begin{definition}
    \label{def-vector}
    由$n$个数$a_1,\cdots,a_n$顺序构成的$n$元有序数组称为$n$元向量。记作
    \begin{equation}
        \alpha=(a_1,a_2,\cdots,a_n),\label{eq-vertor-1}
    \end{equation}
    $a_i$称为向量$\alpha$的第$a_i,i=1,2,\cdots,n$个分量。(\ref{eq-vertor-1})的形式称为行向量,除了行向量形式之外,向量还可以写成列向量的形式
    \begin{eqnarray}
        \alpha=\begin{bmatrix}
            a_1\\\vdots\\a_n
        \end{bmatrix}=(a_1,\cdots,a_n)^T.
    \end{eqnarray}
\end{definition}

向量的相等、加法、减法、数乘等操作参照矩阵的对应运算,不做赘述。

\subsection{向量的线性相关性}

\begin{definition}
    \label{def-linear-composition}
    设$\alpha_1,\cdots,\alpha_m$是$m$个$n$元向量,$k_1,\cdots,k_m$是任意$m$个数,称下列向量
    \[ \beta=k_1\alpha_1+\cdots+k_m\alpha_m \]
    是向量组的$\alpha_1,\cdots,\alpha_m$的一个线性组合。此时,也称向量$\beta$可由向量组$\alpha_1,\cdots,\alpha_m$线性表出。
\end{definition}

\begin{definition}
    \label{def-linear-relation}
    设$\alpha_1,\cdots,\alpha_m$是$m$个$n$元向量,若存在$m$个不全为零的数$k_1,\cdots,k_m$使得
    \begin{equation}
        k_1\alpha_1+\cdots+k_m\alpha_m=0\label{eq-linear-dependence}
    \end{equation}
    则称向量组$\alpha_1,\cdots,\alpha_m$线性相关,否则称为线性无关。
\end{definition}

\begin{remark}
    平行(共线)是描述两个向量之间关系的概念,而共面是描述多个向量之间关系的概念,而线性相关(线性无关)则是一个更高层次的概念。
\end{remark}

\begin{remark}
    线性相关性是从共线、共面等概念引申出来的应用于更高维向量的概念。在二维空间中,向量$\gamma,\eta$的共线即表示存在两个不全为零的实数$k_1,k_2$使得$k_1\gamma+k_2\eta=0$(这里隐含地说明了为什么零向量共线于任何向量);在三维空间中,向量$\gamma,\eta,\varepsilon$共面即表示存在三个不全为零的实数$k_1,k_2,k_3$使得$k_1\gamma+k_2\eta+k_3\varepsilon=0$。
\end{remark}

基于线性相关和线性无关的定义,可以到处一些基本结论(证明从略)。

\begin{thm}
    \label{thm-iff-vector-group-dependence}
    向量组$\alpha_1,\cdots,\alpha_m$线性相关当且仅当至少存在$\alpha_i,1\leqslant i\leqslant m$使可以由其余的向量线性表出。
\end{thm}

\begin{thm}
    \label{thm-independence-vec-grp-append}
    向量组$\alpha_1,\cdots,\alpha_m$线性无关,向量组$\beta,\alpha_1,\cdots,\alpha_m$,则$\beta$可以由$\alpha_1,\cdots,\alpha_m$线性表出,且表示法唯一。
\end{thm}

\begin{definition}
    \label{def-linear-represent}
    设$\alpha_1,\cdots,\alpha_s$与$\beta_1,\cdots,\beta_t$是两组$n$元向量,若每个$\alpha_i,i=1,2,\cdots,s$均可以由$\beta_1,\cdots,\beta_t$线性表出,则称向量组$\alpha_1,\cdots,\alpha_s$可以由向量组$\beta_1,\cdots,\beta_t$线性表出。若向量组$\alpha_1,\cdots,\alpha_s$与$\beta_1,\cdots,\beta_t$可以相互线性表出,则称$\alpha_1,\cdots,\alpha_s$与$\beta_1,\cdots,\beta_t$等价,记作
    \[ \{\alpha_1,\cdots,\alpha_s\}\cong\{\beta_1,\cdots,\beta_t\} \]
\end{definition}

向量组的等价是一种等价关系,拥有自反性、对称性和传递性。

关于向量组线性相关性,有一个重要的充分定理。

\begin{thm}
    \label{thm-important-linear-dependence}
    设$\alpha_1,\cdots,\alpha_s$是一组$n$元向量。若存在另一组$n$元向量$\beta_1,\cdots,\beta_t$有
    \begin{enumerate}
        \item $\alpha_1,\cdots,\alpha_s$可以由向量组$\beta_1,\cdots,\beta_t$线性表出;
        \item $s>t$,
    \end{enumerate}
    则向量组$\alpha_1,\cdots,\alpha_s$线性相关。
\end{thm}

定理的逆否推论如下

\begin{corollary}
    \label{col-important-linear-independence}
    设$\alpha_1,\cdots,\alpha_s$与$\beta_1,\cdots,\beta_t$是两组$n$元向量,若存在
    \begin{enumerate}
        \item $\alpha_1,\cdots,\alpha_s$可以由向量组$\beta_1,\cdots,\beta_t$线性表出;
        \item $\alpha_1,\cdots,\alpha_s$线性无关,
    \end{enumerate}
    则$s\leqslant t$。
\end{corollary}

定理\ref{thm-important-linear-dependence}的证明如下,由于定理\ref{col-important-linear-independence}是前者的逆否,只要证明了前者,则后者正确性显然。证明如下。

\begin{proof}
    \label{proof-important-linear-dependence}

    由条件1可知,存在$c_{ij},i=1,\cdots,s;j=1,\cdots,t$使得
    \[ \alpha_i=c_{i1}\beta_1+\cdots+c_{it}\beta_t \]
    设
    \[ \gamma_i=(c_{i1},\cdots,c_{it}),i=1,\cdots,s \]
    则$\gamma_1,\cdots,\gamma_s$为$s$个$t$元向量。

    由于$s>t$,根据引理\ref{lenma-vector-linear-dependence},$\gamma_1,\cdots,\gamma_s$线性相关,存在不全为零的数$k_1,\cdots,k_s$使得
    \[ k_1\gamma_1+\cdots+k_s\gamma_s=0 \]
    即
    \[ k_1c_{1j}+\cdots+k_sc_{sj}=\sum_{i=1}^sk_ic_{ij}=0,j=1,\cdots,s \]
    又因为
    \begin{equation*}
        \begin{aligned}
            \sum_{i=1}^sk_i\alpha_i
            =& \sum_{i=1}^sk_i\left(\sum_{j=1}^tc_{ij}\beta_j\right) \\
            =& \sum_{i=1}^s\sum_{j=1}^tk_ic_{ij}\beta_j \\
            =& \sum_{j=1}^t\sum_{i=1}^sk_ic_{ij}\beta_j \\
            =& \sum_{j=1}^t0\beta_j=0
        \end{aligned}
    \end{equation*}
    且$k_1,\cdots,k_t$不全为零,所以$\alpha_1,\cdots,\alpha_s$线性相关。
\end{proof}

\begin{lemma}
    \label{lenma-vector-linear-dependence}
    $m$个$n$元向量($m>n$)线性相关。
\end{lemma}

引理\ref{lenma-vector-linear-dependence}的证明就是将向量线性组合展开成为方程组,观察到方程组未知数个数大于方程个数,直接得出方程组存在非零解,从而认定,向量线性相关。

\section{向量组的秩}

\subsection{向量组的秩}

对于线性相关的向量组,秩可以描述其线性相关性的强弱程度。由此我们提出秩的概念。

\begin{definition}
    \label{def-vector-group-rank}
    设向量组$\alpha_1,\cdots,\alpha_m$有$m$个$n$元向量。其中存在$r$个向量线性无关,但是任意的$r+1$个向量都线性相关,则称向量组$\alpha_1,\cdots,\alpha_m$的值为$r$,记作$r\{\alpha_1,\cdots,\alpha_m\}=r$。
\end{definition}

\begin{remark}
    向量组的秩为$r$并不意味着向量组中任取$r$个向量都线性无关,例如向量组
    \[
        a=(1,1,0,0),b=(1,1,0,0),c=(0,1,0,0),d=(0,1,0,0)
    \]
    其秩为$2$,但是若取部分组$a,b$,显然不是线性无关的。
\end{remark}

向量组的秩可以来衡量向量组的线性相关性,通过其定义,容易给出如下充要条件。

\begin{thm}
    \label{thm-vector-group-dependence-rank}
    向量组$\alpha_1,\cdots,\alpha_m$线性相关当且仅当$r\{\alpha_1,\cdots,\alpha_m\}<m$。
\end{thm}

为了确定向量组的秩,我们引入一些相关的概念。

\begin{definition}
    \label{def-max-independence-group}
    设向量组$\alpha_1,\cdots,\alpha_m$的秩为$r$,则$\alpha_1,\cdots,\alpha_m$中任意$r$个线性无关的向量都称为其极大线性无关部分组,简称极大无关组。
\end{definition}

定义了极大无关组之后,我们可以证明向量组的所有极大无关组都是等价的。

\begin{property}
    \label{property-equvilent-mig}
    向量组与其任意一极大无关组等价。
\end{property}

\begin{proof}
    设向量组$\alpha_1,\cdots,\alpha_m$的极大无关组为$\alpha_{i_1},\cdots,\alpha_{i_r}$,显然后者可以被前者线性表出,只需证明前者可以被后者线性表出即可。

    对于向量组$\alpha_1,\cdots,\alpha_m$中的任意向量$\alpha_k$,若$\alpha_k\in\{\alpha_{i_1},\cdots,\alpha_{i_r}\}$,那么可以被其线性标出。否则,向量组$\{\alpha_k,\alpha_{i_1},\cdots,\alpha_{i_r}\}$线性相关,根据定理\ref{thm-independence-vec-grp-append},$\alpha_k$被向量组$\alpha_{i_1},\cdots,\alpha_{i_r}$唯一地线性标出。所以$\alpha_1,\cdots,\alpha_m$可以被$\alpha_{i_1},\cdots,\alpha_{i_r}$线性表出。

    故向量组与其任意一极大无关组等价。
\end{proof}

更进一步地,我们有如下定理。

\begin{thm}
    \label{thm-rank-decrease-by-linear-composition}
    若向量组$\alpha_1,\cdots,\alpha_s$可以由$\beta_1,\cdots,\beta_t$线性表出,则
    \[
        r\{\alpha_1,\cdots,\alpha_s\}\leqslant r\{\beta_1,\cdots,\beta_t\}
    \]
\end{thm}

\begin{proof}
    设向量组$\alpha_1,\cdots,\alpha_s;\beta_1,\cdots,\beta_t$的极大无关组分别为$\alpha_{i_1},\cdots,\alpha_{i_r}$和$\beta_{i_1},\cdots,\beta_{i_p}$。根据定理\ref{property-equvilent-mig},从而有
    \[
        \begin{aligned}
            \{\alpha_{i_1},\cdots,\alpha_{i_r}\}\cong&\{\alpha_1,\cdots,\alpha_s\} \\
            \{\beta_1,\cdots,\beta_t\}\cong&\{\beta_{i_1},\cdots,\beta_{i_p}\}
        \end{aligned}
    \]

    而$\beta_1,\cdots,\beta_t$又可以线性表出$\alpha_1,\cdots,\alpha_s$,所以$\beta_{i_1},\cdots,\beta_{i_p}$线性表出$\alpha_{i_1},\cdots,\alpha_{i_r}$,根据推论\ref{col-important-linear-independence}有$r\leqslant p$,从而有
    \[
        r\{\alpha_1,\cdots,\alpha_s\}\leqslant\{\beta_1,\cdots,\beta_t\}
    \]
\end{proof}

极大无关组拥有如下的等价定义。由此极大无关组的定义可以参见定义\ref{def-max-independence-group}和定义\ref{def-max-independence-group-2}。

\begin{definition}
    \label{def-max-independence-group-2}
    设$\alpha_{i_1},\cdots,\alpha_{i_r}$是向量组$\alpha_1,\cdots,\alpha_m$的一个部分组,若$\alpha_{i_1},\cdots,\alpha_{i_r}$线性无关且每个$\alpha_j(j=1,2,\cdots,m)$均可以由$\alpha_{i_1},\cdots,\alpha_{i_r}$线性表出,则$\alpha_{i_1},\cdots,\alpha_{i_r}$是向量组$\alpha_1,\cdots,\alpha_m$的极大无关组。
\end{definition}

等价性证明容易给出。只需要证明对于一个向量组,根据定义\ref{def-max-independence-group}和定义\ref{def-max-independence-group-2}定义出向量组总是相等的即可。

\begin{proof}
    设向量组$\alpha_1,\cdots,\alpha_m$,根据定义\ref{def-max-independence-group}和定义\ref{def-max-independence-group-2}定义出的极大无关组分别是$V_1,V_2$,根据定义\ref{def-max-independence-group-2},显然有$V_2\subseteq V_1$。反之,根据定义\ref{def-max-independence-group}和性质\ref{property-equvilent-mig},也能有$V_1\subseteq V_2$。所以有$V_1=V_2$,故定义\ref{def-max-independence-group}和定义\ref{def-max-independence-group-2}等价。
\end{proof}

\subsection{向量组的秩与矩阵秩的关系}

对于矩阵可以按照行向量和列向量的方式进行划分,即

\[
    A=[\gamma_i,\cdots,\gamma_n]=\begin{bmatrix}
        \alpha_1\\\vdots\\\alpha_m
    \end{bmatrix}
\]

为了说明两者之间的关系,我们一步一步来。对于矩阵的秩与其行向量的秩,我们有如下关系

\begin{thm}
    \label{thm-rank-matrix-row-vector}
    阶梯形矩阵的秩等于其行向量组的秩。
\end{thm}

\begin{proof}
    取任意一个秩为$r$的$m\times n$的阶梯形矩阵$B$如下
    \[
        B=\begin{bmatrix}
            b_{11} & b_{12} & \cdots & b_{1r} & \cdots & b_{1n} \\
            0      & b_{22} & \cdots & b_{2r} & \cdots & b_{2n} \\
            \vdots & \vdots &        & \vdots &        & \vdots \\
            0      & 0      & \cdots & b_{rr} & \cdots & b_{rn} \\
            0      & 0      & \cdots & 0      & \cdots & 0      \\
            \vdots & \vdots &        & \vdots &        & \vdots \\
            0      & 0      & \cdots & 0      & \cdots & 0      \\
        \end{bmatrix}
    \]
    其中$b_{ii}.i=1,2,\cdots,r$为非零主元。设
    \[
        \gamma_i=\left\{
        \begin{aligned}
            &(0,\cdots,0,b_{ii},b_{i,i+1},b_{in}),i=1,2,\cdots,r\\
            &(0,0,\cdots,0),i=r+1,\cdots,m
        \end{aligned}
        \right.
    \]
    为矩阵$B$的行向量。那么向量组$\gamma_1,\cdots,\gamma_m$的极大无关组显然为$\gamma_1,\cdots,\gamma_r$,秩为$r$。

    故,阶梯形矩阵的秩等于其行向量组的秩。
\end{proof}

定理\ref{thm-rank-matrix-row-vector}证明了阶梯形矩阵的秩与其对应的行向量组的秩之间的关系。由于阶梯形矩阵是由原矩阵通过初等行变换而来,所以我们接着说明矩阵的初等行变换与其对应的行向量组的秩之间的关系。

\begin{thm}
    \label{thm-rank-elem-row-trans-to-vec-grp}
    矩阵的初等行变换不改变行向量的秩。
\end{thm}

\begin{proof}
    $m\times n$矩阵$A$的行向量组为$\gamma_1,\cdots,\gamma_m$,做一次初等行变换为矩阵$B$的行向量组为$\beta_1,\cdots,\beta_m$。显然,$\beta_1,\cdots,\beta_m$可以被$\gamma_1,\cdots,\gamma_m$线性表出,根据定理\ref{thm-rank-decrease-by-linear-composition},有
    \[
        r\{\gamma_1,\cdots,\gamma_m\}\geqslant r\{\beta_1,\cdots,\beta_m\}
    \]
    由于初等行变换可逆,故存在初等行变换可以将$B$还原为$A$,同理
    \[
        r\{\gamma_1,\cdots,\gamma_m\}\leqslant r\{\beta_1,\cdots,\beta_m\}
    \]

    综上所述
    \[
        r\{\gamma_1,\cdots,\gamma_m\}=r\{\beta_1,\cdots,\beta_m\}
    \]
\end{proof}

根据如下定理,我们可以认定,矩阵的秩等于其行向量组的秩,也等于其列向量组的秩。

\begin{thm}
    \label{thm-matrix-rank-is-vec-grp-rank}
    矩阵的秩等于其行向量组的秩,也等于其列向量组的秩。
\end{thm}

\begin{proof}
    对于任意$m\times n$矩阵$A$,其行向量组为$\alpha_1,\cdots,\alpha_n$。矩阵$A$经过初等行变换变换为阶梯形矩阵$B$,其行向量组为$\beta_1,\cdots,\beta_n$。根据定理\ref{thm-rank-matrix-row-vector},$r(B)=r\{\beta_1,\cdots,\beta_n\}$。由于初等行变换不改变矩阵的秩和行向量组的秩(定理\ref{rank-conservation}和定理\ref{thm-rank-elem-row-trans-to-vec-grp}),所以$r(A)=r\{\alpha_1,\cdots,\alpha_n\}$。

    设矩阵$A$的列向量为$\gamma_1,\cdots,\gamma_m$,同理有$r(A^T)=r\{\gamma_1,\cdots,\gamma_m\}$。根据定理\ref{rank-property},从而有
    \[ r(A)=r(A^T)=r\{\gamma_1,\cdots,\gamma_m\} \]

    由此得出结论,矩阵的秩等于其行向量组的秩,也等于其列向量组的秩。
\end{proof}

容易得出如下的定理。

\begin{thm}
    \label{thm-matrix-invertiable-iff-vec-independence}
    设$A$为方阵,则$A$是可逆矩阵的充分必要条件为$A$的行(列)向量组线性无关。
\end{thm}

一种求向量组的极大无关组的方法如下。

设一组$n$元列向量$\alpha_1,\cdots,\alpha_m$,以其为列构造矩阵$A$,将$A$用初等行变换变换为阶梯型矩阵$B$

\[
    B=\begin{bmatrix}
        0 & \cdots & 0 & b_{1j_1} & \cdots & b_{1j_2} & \cdots & b_{1j_r} & \cdots & b_{1m} \\
        0 & \cdots & 0 & 0 & \cdots & b_{2j_2} & \cdots & b_{2j_r} & \cdots & b_{2m} \\
        \vdots&&\vdots & \vdots && \vdots &        & \vdots &        & \vdots \\
        0 & \cdots & 0 & 0 & \cdots & 0      & \cdots & b_{rj_r} & \cdots & b_{rm} \\
        0 & \cdots & 0 & 0 & \cdots & 0      & \cdots & 0      & \cdots & 0      \\
        \vdots&&\vdots & \vdots && \vdots &        & \vdots &        & \vdots \\
        0 & \cdots & 0 & 0 & \cdots & 0      & \cdots & 0      & \cdots & 0      \\
    \end{bmatrix}
\]

其中$b_{1j_1},\cdots,b_{rj_r}$为$B$的非零主元,可以断言向量组$\alpha_1,\cdots,\alpha_m$的秩为$r$,且$\alpha_{j_1},\cdots,\alpha_{j_r}$是一个极大无关组。这是显然的,因为以$\alpha_{j_1},\cdots,\alpha_{j_r}$为列向量构造的矩阵$A'$在同样的初等行变换得到的阶梯形矩阵$B'$为

\[
    B'=\begin{bmatrix}
        b_{1j_1} & b_{1j_2} & \cdots & b_{1j_r} \\
        0 & b_{2j_2} & \cdots & b_{2j_r} \\
        \vdots & \vdots &  & \vdots \\
        0 & 0 & \cdots & b_{rj_r} \\
        0 & 0 & \cdots & 0 \\
        \vdots & \vdots & & \vdots \\
        0 & 0 & \cdots & 0
    \end{bmatrix}
\]

其中$b_{1j_1},\cdots,b_{rj_r}$为$B'$的主元,由此得到向量组$\alpha_{j_1},\cdots,\alpha_{j_r}$秩为$r$,线性无关。重新整理一下证明思路,由于$B$的秩为$r$,所以我们可以得出向量组$\alpha_1,\cdots,\alpha_m$的秩为$r$(矩阵的秩等于行、列向量组的秩),为了找到向量组$\alpha_1,\cdots,\alpha_m$中的极大无关组,我们按照阶梯形矩阵的特征挑选对应的向量,这些向量可以证明为线性无关的,所以挑选出的向量组$\alpha_{j_1},\cdots,\alpha_{j_r}$就是线性无关组。

\begin{thm}
    \label{thm-mat-rank-decrease-through-multiplication}
    设$A$为$m\times p$矩阵,$B$为$p\times n$矩阵,则
    \[ r(AB)=\min\{r(A),r(B)\} \]
\end{thm}

可以利用定理\ref{thm-rank-decrease-by-linear-composition}证明,证明细节从略。

\subsection{齐次线性方程组的解}

对于$m\times n$的矩阵$A$按照列分块为$A=[\alpha_1,\cdots,\alpha_n]$,齐次线性方程组

\begin{equation}
    AX=0\label{eq-equations}
\end{equation}

可以表示为

\begin{equation}
    x_1\alpha_1+\cdots+x_n\alpha_n=0\label{eq-equation-vec}
\end{equation}

称(\ref{eq-equation-vec})为(\ref{eq-equations})的向量表达式。于是齐次方程组(\ref{eq-equations})存在非零解的充分必要条件是$\alpha_1,\cdots,\alpha_n$线性相关。

\begin{thm}
    \label{thm-equation-nonzero-solve}
    齐次线性方程组$AX=0$有非零解的充分必要条件是$r(A)$小于$A$的列数。也可以等价叙述为齐次线性方程组$AX=0$只有零解当且仅当$r(A)$等于$A$的列数。
\end{thm}

我们把使得方程组成立的数看作向量,称为解向量。

\begin{property}
    \label{property-solution-vector}
    设$X_1,X_2$为齐次线性方程组$AX=0$的两个任意解,$k_1$为任意常数,那么
    \begin{enumerate}
        \item $X_1+X_2$是该方程组的解向量;
        \item $k_1X_1$是该方程组的解向量。
    \end{enumerate}
\end{property}

证明从略。

在明确了解向量的基础上,我们定义齐次线性方程组基础解系。

\begin{definition}
    \label{def-solution-base}
    设$X_1,\cdots,X_t$为齐次线性方程组$AX=0$的$t$个解向量,若
    \begin{enumerate}
        \item $X_1,\cdots,X_t$线性无关;
        \item 任意该方程的解向量都可以由$X_1,\cdots,X_t$线性表出。
    \end{enumerate}
    则称$X_1,\cdots,X_t$是该方程组的一个基础解系。
\end{definition}

从基础解系的角度来说,沿用\ref{def-solution-base}中的符号,该方程组的一般解可以表示为
\[
    k_1X_1+\cdots+k_tX_t
\]
其中$k_1,\cdots,k_t$为任意常数。

以下定理说明了基础解系中解向量的数量和求解问题。

\begin{thm}
    \label{thm-solution-base-calculation}
    设$A$是$m\times n$矩阵。若$r(A)=r<n$,则齐次线性方程组$AX=0$存在基础解系,其中包含$n-r$个解向量。
\end{thm}

对于齐次线性方程组,通过初等行变换变化为阶梯形,将所有非主元元素所在的列移动到等式的右边,依次取$1$(其余取$0$),得到$n-r$个向量组成的基础解系。

证明从略。

\begin{example}
    设$A$为$m\times n$矩阵,且$r(A)=r<n$,则齐次线性方程组$AX=0$的任意$n-r$个线性无关的解向量都构成有一个基础解系。
\end{example}

\begin{proof}
    设其基础解系为$X_1,\cdots,X_{n-r}$,任意$n-r$个线性无关的解向量分别为$X_1^*,\cdots,X_{n-r}^*$。对于任意一个解向量$X_0$,$\{X_0,X_1^*,\cdots,X_{n-r}^*\}$可以由$X_1,\cdots,X_{n-r}$线性表出,根据定理\ref{thm-important-linear-dependence},$\{X_0,X_1^*,\cdots,X_{n-r}^*\}$必然线性相关,又因为$X_1^*,\cdots,X_{n-r}^*$线性无关,根据定理\ref{thm-independence-vec-grp-append},$X_0$可以被$X_1^*,\cdots,X_{n-r}^*$唯一线性表出。
\end{proof}

\subsection{非齐次线性方程组解的结构}

设$A$是$m\times n$矩阵,对$A$按列分块$A=[\alpha_1,\cdots,\alpha_n]$,则非齐次线性方程组
\[
    AX=b
\]
可以表示为
\[
    x_1\alpha_1+\cdots+x_n\alpha_n=\beta
\]
其中$\beta$为列矩阵$b$,后者成为前者的向量表达式。易知该方程组有解的充分必要条件是$\beta$可以被$\alpha_1,\cdots,\alpha_n$线性表出,这等价于下式成立
\[
    r\{\alpha_1,\cdots,\alpha_n\}=r\{\beta,\alpha_1,\cdots,\alpha_n\}
\]

以下这个定理说明了这个道理。

\begin{thm}
    \label{thm-not-homo-equations-solvable}
    设$A$是$m\times n$矩阵,$b$为$m\times 1$矩阵,则非齐次线性方程组$AX=b$有解的充分必要条件为$r(A)=r(\tilde{A})$,其中$\tilde{A}=[A,b]$。
\end{thm}

结合定理\ref{thm-independence-vec-grp-append}可以得出如下定理

\begin{thm}
    \label{thm-not-homo-equations-single-solution}
    设$A$是$m\times n$矩阵,$b$为$m\times 1$矩阵,则非齐次线性方程组$AX=b$有唯一解的充分必要条件为$r(A)=r(\tilde{A})=n$,这里$\tilde{A}=[A,b]$。

    等价叙述为$AX=b$存在无穷多解当且仅当$r(A)=r(\tilde{A})<n$。
\end{thm}

对于非齐次线性方程组$AX=b$我们称$AX=0$为其的导出方程组。对于非齐次线性方程组与其导出方程组,有如下容易证明的性质。

\begin{property}
    \label{property-solution-homo-and-nohomo-equations}
    \begin{enumerate}
        \item 非齐次线性方程组的任意两个解的差为其导出方程组的解;
        \item 非齐次线性方程组的解与其导出方程组的解的和仍然是该非齐次线性方程组的解。
    \end{enumerate}
\end{property}

非齐次线性方程组的解的结构不同于齐次线性方程组,但是两者是类似的。依托性质\ref{property-solution-homo-and-nohomo-equations}可以得出如下定理

\begin{thm}
    \label{thm-solution-struct-not-homo-equation}
    设非齐次线性方程组$AX=b$拥有无穷个解,其一般形式为
    \begin{equation}
        X_0+k_1X_1+k_2X_2+\cdots+k_tX_t
        \label{eq-solution-struct-not-homo-equation}
    \end{equation}
    其中$X_0$为$AX=b$的一个解,$X_1,\cdots,X_t$为导出方程组$AX=0$的基础解系,$k_1,\cdots,k_t$为任意常数。
\end{thm}

\begin{proof}
    显然\ref{eq-solution-struct-not-homo-equation}是$AX=b$的解。反之对于该方程组的任意一个解$X$,有$X-X_0$为$AX=0$的解,可以表示为$X_1,\cdots,X_t$的线性组合,不妨设为$k_1X_1+\cdots+k_tX_t$,那么显然有$X=X_0+k_1X_1+\cdots+k_tX_t$。

    综上所述,(\ref{eq-solution-struct-not-homo-equation})确为$AX=b$的一般解。
\end{proof}
