\chapter{矩阵}

\section{矩阵的秩与初等变换}

\subsection{矩阵的秩}

\begin{thm}
    \label{rank-conservation}
    一个矩阵用初等行变换化成的的阶梯型矩阵中,其中非零行的数量唯一确定。
\end{thm}

定理\ref{rank-conservation}说明了初等行变换不改变矩阵的秩,注意这里没有说明初等列变换不改变矩阵的秩,这条性质是通过相抵标准型证明的。(秩的定义见\ref{rank})

\begin{definition}
    \label{rank}
    矩阵$A$用初等行变换化成的阶梯型矩阵中非零行的数目称为矩阵的秩,记作$r(A)$。
\end{definition}

\subsection{矩阵的初等变换}

矩阵的列变换方式参照矩阵的行变换,两者并称矩阵的初等变换。这里注意一点,我们目前只说明了行变换不改变矩阵的秩,但是并没有说明同时作用行变换和列变换或者单独作用列变换的时候矩阵的秩的不变性。

在引入了列变换的基础上我们可以定义矩阵相抵。

\begin{definition}
    \label{xiangdi} % 找不到英文啊
    如果矩阵$A$可以通过有限次初等变换化为$B$,则称$A$相抵于$B$,记作$A\cong B$。
\end{definition}

容易证明定理\ref{rank-and-xiangdi}。

\begin{thm}
    \label{rank-and-xiangdi}
    如果$m\times n$的矩阵$A$有$r(A)=r$,那么矩阵$A$相抵于
    $$
    \tilde{B}=\begin{bmatrix}
        1 & 0 & \cdots & 0 & \cdots & 0 \\
        0 & 1 & \cdots & 0 & \cdots & 0 \\
        \vdots & \vdots & & \vdots & & \vdots \\
        0 & 0 & \cdots & 1 & \cdots & 0 \\
        0 & 0 & \cdots & 0 & \cdots & 0 \\
        \vdots & \vdots & & \vdots & & \vdots \\
        0 & 0 & \cdots & 0 & \cdots & 0 \\
    \end{bmatrix}_{m\times n}
    $$
    上矩阵中含有$1$的行为前$r$行。$\tilde{B}$称为矩阵$A$的相抵标准型。
\end{thm}
需要注意一点是定理\ref{rank-and-xiangdi}只说明了矩阵$A$相抵于$\tilde{B}$,并没有排除矩阵$A$相抵于其他非$\tilde{B}$的相抵标准型矩阵的可能性,即这条定理并没有说明相抵标准型的唯一性。

\subsection{初等矩阵}

\begin{definition}
    \label{elementory-matrix}
    单位矩阵经过一次初等变换得到的矩阵称为初等矩阵。
\end{definition}

不难证明定理\ref{elementory-transformation-and-matrix}。

\begin{thm}
    \label{elementory-transformation-and-matrix}
    对于$m\times n$的矩阵$A$,有
    \begin{enumerate}
        \item 做一次初等行变换等同于左乘一个对应的$m$阶初等矩阵;
        \item 做一次初等列变换等同于右乘一个对应的$n$阶初等矩阵;
    \end{enumerate}
\end{thm}

对于初等矩阵,有如下的重要性质。

\begin{property}
    \label{elementory-matrix-property}
    \begin{enumerate}
        \item 初等矩阵满秩且初等矩阵的乘积也是满秩的;
        \item 对于任意初等矩阵$P$存在初等矩阵$Q$使得$PQ=QP=I$。
    \end{enumerate}
\end{property}

其中性质\ref{elementory-matrix-property}的1证明如下。
\begin{proof}
    \label{proof-1}
    初等矩阵满秩显然。

    设$s$个初等矩阵$P_1,\cdots P_s$,令$P=P_1\cdots P_s$,从而有$P=P_1\cdots P_sI$,可以看作单位矩阵经过$s$次初等行变换得到矩阵$P$,显然$P$满秩。
\end{proof}

\begin{remark}
    可以注意的一点是,证明\ref{proof-1}中无论矩阵$P_i,i=1,\cdots,s$对应的是初等行变换还是初等列变换,都可以等价为初等行变换。即初等行变换和列变换的矩阵的形式是一致的,产生本质区别的是左乘还是右乘。
\end{remark}

由满秩矩阵的定义很容易导出定理\ref{full-rank-matrix-and-elementory-matrix}。

\begin{thm}
    \label{full-rank-matrix-and-elementory-matrix}
    满秩矩阵可以表示成若干初等矩阵的乘积。
\end{thm}

进一步地,我们可以推导出一个关于矩阵相抵的充分必要条件如定理\ref{iff-matrix-xiangdi}。

\begin{thm}
    \label{iff-matrix-xiangdi}
    有两个$m\times n$矩阵$A,B$,那么$A\cong B$当且仅当存在$m$阶满秩矩阵$P$与$n$阶满秩矩阵$Q$使得$PAQ=B$。
\end{thm}

在以上的定理完整之后,我们可以给出相抵标准型唯一性的证明,即证明若矩阵$A\cong \tilde{B}_1\wedge A\cong\tilde{B}_2$,其中$\tilde{B}_1,\tilde{B}_2$为相抵标准型则必然有$\tilde{B}_1=\tilde{B}_2$。定理\ref{xiangdi-singular}说明了这一点。

\begin{thm}
    \label{xiangdi-singular}
    同型矩阵$A,B$相抵当且仅当$r(A)=r(B)$。
\end{thm}

定理\ref{xiangdi-singular}证明如下。

\begin{proof}
    \label{pf-xiangdi-singular}
    充分性显然。

    必要性证明。设$A\cong B,r(A)=s,r(B)=t$,$A,B$的相抵标准型分别为$C_1,C_2$。于是有$C_1\cong C_2$,那么存在初等矩阵$P_1,\cdots,P_k,Q_1,\cdots,Q_l$使得
    $$
    P_k\cdots P_1C_1Q_1\cdots Q_l=C_2
    $$
    从而对于$P_i,i=1,\cdots,k$有$P_i'$使得$P_i' P_i=I$,故
    $$
    C_1Q_1\cdots Q_l=P_1'\cdots P_k'C_2
    $$
    令$Q=Q_1\cdots Q_l,P=P_1'\cdots P_k'$,则$P,Q$满秩且
    $$
    PC_2=\begin{bmatrix}
        P^* & 0
    \end{bmatrix}=C_1Q=\begin{bmatrix}
        Q^* \\ 0
    \end{bmatrix}
    $$
    可以推出$p_{ij}=0,i=s+1,\cdots,m;j=1,\cdots,t$且$q_{ij}=0,i=1,\cdots,s;j=t+1,\cdots,n$。此时若假设$r(A)\neq r(B)$,不失一般性,我们假定$t>s$,此时矩阵
    $$
    P=\begin{bmatrix}
        p_{11} & \cdots & p_{1t} & p_{1,t+1} & \cdots & p_{1m} \\
        \vdots & & \vdots & \vdots & & \vdots \\
        p_{s1} & \cdots & p_{st} & p_{s,t+1} & \cdots & p_{sm} \\
        0 & \cdots & 0 & p_{s+1,t+1} & \cdots & p_{s+1,m} \\
        \vdots & & \vdots & \vdots & & \vdots \\
        0 & \cdots & 0 & p_{m,t+1} & \cdots & p_{mm} \\
    \end{bmatrix}
    $$
    矩阵可以分为前$s$行和后$m-s$行,对$P$变换为阶梯型$P'$,那么前$s$行变换后得到的非零行的数量小于等于$s$,后$m-s$行经过变换得到的非零行的数量小于等于$\min\{m-s,m-t\}=m-t$行,故$P'$的非零行的数量小于等于$s+m-t$行,由于初等行变换不改变矩阵的秩,所以$r(P)=r(P')\leqslant s+m-t<t+m-t=m$,即$P$降秩,这与$P$的构造矛盾。
\end{proof}

\begin{remark}
    由此我们知道,初等变换不改变矩阵的秩,即初等行变换和初等列变换都不改变矩阵的秩。
\end{remark}

在定理\ref{pf-xiangdi-singular}的基础上,我们可以推导出定理\ref{rank-property}。

\begin{thm}
    \label{rank-property}
    对于$m\times n$的矩阵$A$,有
    \begin{enumerate}
        \item $r(A)=r(A^T)$;
        \item $r(A)=r(PA)=r(AQ)=r(PAQ)$,其中$P,Q$分别是$m,n$阶的满秩矩阵。
    \end{enumerate}
\end{thm}



\section{可逆矩阵} % Klee matrix

\begin{definition}
    \label{inversable-matrix}
    设$A$为$n$阶方阵,若存在同型的方阵$B$满足
    \[ AB=BA=I, \]
    那么就称$A$为可逆矩阵,$B$为$A$的逆矩阵,记作$B=A^{-1}$。
\end{definition}

由于方阵的乘法可以构成群,故根据离散数学中关于群的叙述,如果存在$B_1,B_2$方阵有$B_1A=AB_2=I$,那么必然有$B_1=B_2$。

对于前文关于初等矩阵的叙述,很容易给出每个初等矩阵的逆矩阵如表\ref{inverse-elementory}所示。由此可以得出,所有的满秩矩阵都是可逆矩阵,这是定理\ref{full-rank-iff-inversable}的充分性。

\begin{table}[!hbt]
    \centering
    \begin{tabular}{cccc}
        \toprule
        原矩阵 & $E_i(k)$ & $E_{ij}(k)$ & $E_{ij}$ \\
        \midrule
        逆矩阵 & $E_i\left(\frac1k\right)$ & $E_{ij}(-k)$ & $E_{ji}$ \\
        \bottomrule
    \end{tabular}
    \caption{初等矩阵的逆矩阵}
    \label{inverse-elementory}
\end{table}

\begin{thm}
    \label{full-rank-iff-inversable}
    方阵$A$可逆的充分必要条件是方阵$A$满秩。
\end{thm}

证明定理\ref{full-rank-iff-inversable}的必要性的过程如下。

\begin{proof}
    \label{proof-full-rank-iff-inversable}
    设$A$为可逆且降秩方阵,那么存在$A^{-1}$使得$AA^{-1}=I$。由于$A$降秩,所以存在一系列初等行变换对应的矩阵$P$使得$PA=B$至少包含一个零行,故$PAA^{-1}=BA^{-1}=P$。由于$B$中包含零行,则$BA^{-1}$中必然也包含零行,不可能有$BA^{-1}=P$,故$A$满秩。
\end{proof}

可逆矩阵还具有一些良好的性质。证明是显然的。

\begin{property}
    \label{property-of-inversable-matrix}
    矩阵$A,B$为可逆矩阵,则
    \begin{enumerate}
        \item $(A^{-1})^{-1}=A$
        \item $(A^{-1})^T=(A^T)^{-1}$
        \item $(AB)^{-1}=B^{-1}A^{-1}$
    \end{enumerate}
\end{property}

\section{分块矩阵}

分块矩阵的具体定义不在说明,这里给出利用分块矩阵证明的若干性质\ref{property-block-matrix}和定理\ref{block-matrix-induction}。

\begin{property}
    \label{property-block-matrix}
    \begin{enumerate}
        \item $r\begin{bmatrix}
            A & B \\ C & D
        \end{bmatrix}\geqslant r(A)$
        \item $r\begin{bmatrix}
            A&0\\C&B
        \end{bmatrix}\geqslant\begin{bmatrix}
            A&0\\0&B
        \end{bmatrix}$
        \item $r\begin{bmatrix}
            A&0\\0&B
        \end{bmatrix}=\begin{bmatrix}
            0&A\\B&0
        \end{bmatrix}=r(A)+r(B)$
        \item 若$A,B$均为满秩方阵,则$\begin{bmatrix}
            A&0\\C&B
        \end{bmatrix},\begin{bmatrix}
            A&C\\0&B
        \end{bmatrix}$均满秩
    \end{enumerate}
\end{property}

\begin{thm}
    \label{block-matrix-induction}
    \begin{enumerate}
        \item 若$A,B$为同型的方阵,那么$r(A+B)\leqslant r(A)+r(B)$
        \item 若$A,B$分别为$s\times n,n\times t$的矩阵,那么$r(AB)\geqslant r(A)+r(B)-n$
    \end{enumerate}
\end{thm}

\section{若干特殊矩阵}

\subsection{对称矩阵与反对称矩阵}

\begin{definition}
    \label{symmtric-and-asymmtric-matrix}
    设矩阵$A$为方阵。若$A=A^T$,则称其为对称矩阵;若$A=-A^T$,则称其为反对称矩阵。
\end{definition}

\begin{remark}
    按照定义\ref{symmtric-and-asymmtric-matrix},取任意方阵$A$,必然有$A+A^T$为对称矩阵,$A-A^T,A^T-A$为反对称矩阵。进一步地,矩阵$A$可以表示为对称矩阵和非对称矩阵之和的形式
    \[ A=\frac12(A+A^T)+\frac12(A-A^T). \]
\end{remark}

\subsection{对角矩阵}

\begin{definition}
    \label{diag-matrix}
    若一个方阵除了主对角线之外的所有元素均为$0$,那么称其为对角矩阵。形如
    \[
        \begin{bmatrix}
        a_1 & 0     & \cdots & 0 \\
        0   & a_2   & \cdots & 0 \\
        \vdots & \vdots &    & \vdots \\
        0 & 0 & \cdots & a_n
        \end{bmatrix}  
    \]
    上面的对角矩阵可以简记为$\diag(a_1,\cdots,a_n)$。

    当$a_1=\cdots=a_n=k$时,称其为数量矩阵,单位矩阵是$k=1$时的特例。
\end{definition}

拓展到分块矩阵上,可以定义准对角矩阵(也称为分块对角矩阵)。

\begin{definition}
    \label{block-diag-matrix}
    $A$是分块矩阵
    \[
        \begin{bmatrix}
        A_1 & 0     & \cdots & 0 \\
        0   & A_2   & \cdots & 0 \\
        \vdots & \vdots &    & \vdots \\
        0 & 0 & \cdots & A_n
        \end{bmatrix}  
    \]
    其中子块$A_1,\cdots,A_n$均为方阵,则称$A$为准对角矩阵或分块对角矩阵。
\end{definition}

\begin{remark}
    显然,当$A_1,\cdots,A_n$均为一阶方阵的时候,该准对角矩阵为对角矩阵。
\end{remark}

\begin{remark}
    不难证明,准对角矩阵$A=\diag(A_1,\cdots,A_n)$可逆的充分必要条件为$A_1,\cdots,A_n$可逆。
\end{remark}

\subsection{三角矩阵}

\begin{definition}
    \label{triangle-matrix}
    设方阵$A=[a_{ij}]_{n\times n}$。若$i>j$均有$a_{ij}=0$,则$A$为上三角矩阵;若$i<j$均有$a_{ij}=0$,则$A$为下三角矩阵。上三角矩阵和下三角矩阵合称为三角矩阵。
\end{definition}

\begin{thm}
    \label{triangle-matrix-inversable}
    三角矩阵可逆当且仅当其主对角线元不全为零。
\end{thm}

定理\ref{triangle-matrix-inversable}的证明如下。

\begin{proof}
    \label{proof-triangle-matrix-inversable}
    充分性显然。

    必要性证明如下。不失一般性,我们证明上三角矩阵。设$A$为可逆的上三角矩阵,对其阶数作数学归纳法。显然阶数为$1$的时候成立,假设当阶数为$n-1$的时候必要性也成立,考虑任意一$n$阶可逆上三角矩阵
    \[
        A=\begin{bmatrix}
            a_{11}  &   a_{12}  &   \cdots  &   a_{1n} \\
            0       &   a_{22}  &   \cdots  &   a_{2n} \\
            \vdots  &   \vdots  &   \vdots  &   \vdots \\
            0       &   0       &   \cdots  &   a_{nn}
        \end{bmatrix}
    \]
    
    因为$A$可逆,所以$A$满秩,由此$a_{nn}\neq0$,对$A$分块得到
    \[
        A=\begin{bmatrix}
            A_1&\alpha\\0&a_{nn}
        \end{bmatrix}
    \]
    其中
    \[
        A_1=\begin{bmatrix}
            a_{11}  &   a_{12}  &   \cdots  &   a_{1,n-1} \\
            0       &   a_{22}  &   \cdots  &   a_{2,n-1} \\
            \vdots  &   \vdots  &   \vdots  &   \vdots \\
            0       &   0       &   \cdots  &   a_{n-1,n-1}
        \end{bmatrix}
    \]
    按照相同方式分块$A^{-1}=[b_{ij}]_{n\times n}$
    \[
        A^{-1}=\begin{bmatrix}
            B_1&\beta_1\\\beta_2&b_{nn}
        \end{bmatrix}
    \]
    其中$B_1$为$n-1$阶方阵。

    于是有
    \[
        A^{-1}A=\begin{bmatrix}
            B_1A_1 & B_1\alpha+a_{nn}\beta_1 \\
            \beta_2A_1 & \beta_2\alpha+a_{nn}b_{nn}
        \end{bmatrix}=I_n=\begin{bmatrix}
            I_{n-1} & 0 \\ 0 & 1
        \end{bmatrix}
    \]
    由此,$B_1A_1=I_{n-1}$,根据归纳假设,$A_1$可逆且为上三角矩阵,所以$A_1$的所有主对角元非零。又因为$a_{nn}\neq0$,所以$A$的主对角元均非零。
\end{proof}

利用数学归纳法不难证明定理\ref{triangle-matrix-inversable-seal}。

\begin{thm}
    \label{triangle-matrix-inversable-seal}
    可逆的上(下)三角矩阵的逆矩阵也是一个上(下)三角矩阵。
\end{thm}

关于方阵的LU分解,有如下定理\ref{LU-decomposition}。

\begin{thm}
    \label{LU-decomposition}
    设$A=[a_{ij}]_{n\times n}$,若$A$的顺序主子式$A_k=[a_{ij}]_{k\times k},k=1,2,\cdots,n$均满秩,则$A$可以表示成
    \begin{equation}
        A=LU\label{eq-LU-decomposition}
    \end{equation}
    其中$L$为主对角元均为$1$的$n$阶下三角矩阵,$U$为$n$阶可逆上三角矩阵。
\end{thm}

定理\ref{LU-decomposition}的证明如下。

\begin{proof}
    \label{proof-LU-decomposition}
    由于$A$是满秩的,所以$L,U$均满秩,故式\ref{eq-LU-decomposition}等价于
    \[
        L^{-1}A=U
    \]
    若$L$的主对角元均为$1$,根据定理\ref{triangle-matrix-inversable-seal}不难证明$L^{-1}$的主对角元必然都为$1$,所以定理的证明转换为证明$A$满足上述条件的时候,存在主对角元均为$1$的下三角矩阵$L'$使得$L'A=U$为上三角矩阵,由于$L',A$均可逆,故$U$的可逆性不必赘述。

    对$A$的阶数$n$进行数学归纳法。当$n=1$的时候显然。假设$A$为$n-1$阶矩阵时假设结论成立,以下证明阶数为$n$时结论也成立。

    对$A$分块
    \[
        A=\begin{bmatrix}
            B_1 & \alpha \\ \beta & a_{nn}
        \end{bmatrix}
    \]
    其中$B_1=A_{n-1}$为$n-1$阶方阵,拥有顺序主子式$A_1,\cdots,A_{n-1}$,根据归纳假设,存在主对角元均为一的$n-1$阶下三角矩阵$L_1'$使得$L_1'B_1=U_1$为可逆的上三角矩阵,由于$B_1$可逆,故令
    \[
        L'=\begin{bmatrix}
            L_1' & 0 \\ -\beta B_1^{-1} & 1
        \end{bmatrix}
    \]
    有
    \[
        L'A=\begin{bmatrix}
            L_1'B_1 & L_1'\alpha \\
            -\beta B_1^{-1}B_1+\beta & -\beta B_1^{-1}\alpha+a_{nn}
        \end{bmatrix}=\begin{bmatrix}
            L_1'B_1 & L_1'\alpha \\
            0 & -\beta B_1^{-1}\alpha+a_{nn}
        \end{bmatrix}=U
    \]
    显然,$U$为$n$阶上三角矩阵。
\end{proof}
