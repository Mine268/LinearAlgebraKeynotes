\documentclass[10pt,b5paper,openany]{book}
\usepackage{amsmath}
\usepackage{amssymb}
\usepackage{amsthm}
\usepackage{booktabs}
% \usepackage{mathpple}
% \usepackage{upgreek}
% \usepackage{mathpazo}
\usepackage{lipsum}
\usepackage{cite}
\usepackage[colorlinks,linkcolor=black]{hyperref}
\usepackage{graphicx}
\usepackage{wrapfig}
\usepackage{savesym}
\usepackage{amsfonts}
\usepackage[margin=1.5in]{geometry}
\usepackage{fancybox}
\usepackage{fontspec}

\usepackage{fancyhdr}
\usepackage{epigraph}
\usepackage{caption}
\usepackage[all]{xy}
\usepackage{tikz}
\usepackage{amsmath,amscd}
\usepackage{geometry}
\geometry{right=2.5cm,left=2.5cm,top=2.5cm,bottom=2.5cm}
\pagestyle{fancy} \lhead{mine268}\chead{线性代数keynote}\rhead{copyleft}
\usepackage{amsmath,amscd}
\usepackage{bm}
\usepackage{titlesec}%chapter1修改为第1章
\renewcommand{\chaptername}{第{\thechapter}章}
\titleformat{\chapter}[block]{\Huge\bfseries}{\chaptername}{10pt}{\\}
\usepackage[UTF8]{ctex}
\makeatletter % `@' now normal "letter"
\@addtoreset{equation}{section}
\makeatother  % `@' is restored as "non-letter"
\renewcommand\theequation{\oldstylenums{\thesection}%
                   .\oldstylenums{\arabic{equation}}}

\newcommand{\dd}{\mathrm{d}}
\newcommand{\diag}{\mathrm{diag}}
% Theorm, lemma, proof etc
\newtheorem{lemma}{\indent 引理}[section]
\newtheorem{thm}{\indent 定理}[section]
\newtheorem{definition}{\indent 定义}[section]
\newtheorem{example}{\indent 例}[section]
\newtheorem{property}{\indent 性质}[section]
\newtheorem{remark}{\indent 注解}[section]
\newtheorem{corollary}{\indent 推论}[section]
% \newtheorem{proof}{证明}[section]
\renewcommand{\proofname}{\indent 证明}

\title{线性代数Keynote}
\author{mine268}

\begin{document}
	\maketitle

	记录学习中存在困惑的地方和重要的性质和定理。

	\chapter{矩阵}

\section{矩阵的秩与初等变换}

\subsection{矩阵的秩}

\begin{thm}
    \label{rank-conservation}
    一个矩阵用初等行变换化成的的阶梯型矩阵中,其中非零行的数量唯一确定。
\end{thm}

定理\ref{rank-conservation}说明了初等行变换不改变矩阵的秩,注意这里没有说明初等列变换不改变矩阵的秩,这条性质是通过相抵标准型证明的。(秩的定义见\ref{rank})

\begin{definition}
    \label{rank}
    矩阵$A$用初等行变换化成的阶梯型矩阵中非零行的数目称为矩阵的秩,记作$r(A)$。
\end{definition}

\subsection{矩阵的初等变换}

矩阵的列变换方式参照矩阵的行变换,两者并称矩阵的初等变换。这里注意一点,我们目前只说明了行变换不改变矩阵的秩,但是并没有说明同时作用行变换和列变换或者单独作用列变换的时候矩阵的秩的不变性。

在引入了列变换的基础上我们可以定义矩阵相抵。

\begin{definition}
    \label{xiangdi} % 找不到英文啊
    如果矩阵$A$可以通过有限次初等变换化为$B$,则称$A$相抵于$B$,记作$A\cong B$。
\end{definition}

容易证明定理\ref{rank-and-xiangdi}。

\begin{thm}
    \label{rank-and-xiangdi}
    如果$m\times n$的矩阵$A$有$r(A)=r$,那么矩阵$A$相抵于
    $$
    \tilde{B}=\begin{bmatrix}
        1 & 0 & \cdots & 0 & \cdots & 0 \\
        0 & 1 & \cdots & 0 & \cdots & 0 \\
        \vdots & \vdots & & \vdots & & \vdots \\
        0 & 0 & \cdots & 1 & \cdots & 0 \\
        0 & 0 & \cdots & 0 & \cdots & 0 \\
        \vdots & \vdots & & \vdots & & \vdots \\
        0 & 0 & \cdots & 0 & \cdots & 0 \\
    \end{bmatrix}_{m\times n}
    $$
    上矩阵中含有$1$的行为前$r$行。$\tilde{B}$称为矩阵$A$的相抵标准型。
\end{thm}
需要注意一点是定理\ref{rank-and-xiangdi}只说明了矩阵$A$相抵于$\tilde{B}$,并没有排除矩阵$A$相抵于其他非$\tilde{B}$的相抵标准型矩阵的可能性,即这条定理并没有说明相抵标准型的唯一性。

\subsection{初等矩阵}

\begin{definition}
    \label{elementory-matrix}
    单位矩阵经过一次初等变换得到的矩阵称为初等矩阵。
\end{definition}

不难证明定理\ref{elementory-transformation-and-matrix}。

\begin{thm}
    \label{elementory-transformation-and-matrix}
    对于$m\times n$的矩阵$A$,有
    \begin{enumerate}
        \item 做一次初等行变换等同于左乘一个对应的$m$阶初等矩阵;
        \item 做一次初等列变换等同于右乘一个对应的$n$阶初等矩阵;
    \end{enumerate}
\end{thm}

对于初等矩阵,有如下的重要性质。

\begin{property}
    \label{elementory-matrix-property}
    \begin{enumerate}
        \item 初等举证满秩且初等举证的乘积也是满秩的;
        \item 对于任意初等举证$P$存在初等举证$Q$使得$PQ=QP=I$。
    \end{enumerate}
\end{property}

其中性质\ref{elementory-matrix-property}的1证明如下。
\begin{proof}
    \label{proof-1}
    初等矩阵满秩显然。

    设$s$个初等矩阵$P_1,\cdots P_s$,令$P=P_1\cdots P_s$,从而有$P=P_1\cdots P_sI$,可以看作单位矩阵经过$s$次初等行变换得到矩阵$P$,显然$P$满秩。
\end{proof}

\begin{remark}
    可以注意的一点是,证明\ref{proof-1}中无论矩阵$P_i,i=1,\cdots,s$对应的是初等行变换还是初等列变换,都可以等价为初等行变换。即初等行变换和列变换的举证的形式是一致的,产生本质区别的是左乘还是右乘。
\end{remark}

由满秩矩阵的定义很容易导出定理\ref{full-rank-matrix-and-elementory-matrix}。

\begin{thm}
    \label{full-rank-matrix-and-elementory-matrix}
    满秩矩阵可以表示成若干初等矩阵的乘积。
\end{thm}

进一步地,我们可以推到出一个关于矩阵相抵的充分必要条件如定理\ref{iff-matrix-xiangdi}。

\begin{thm}
    \label{iff-matrix-xiangdi}
    有两个$m\times n$矩阵$A,B$,那么$A\cong B$当且仅当存在$m$阶满秩矩阵$P$与$n$阶满秩矩阵$Q$使得$PAQ=B$。
\end{thm}

在以上的定理完整之后,我们可以给出相抵标准型唯一性的证明,即证明若矩阵$A\cong \tilde{B}_1\wedge A\cong\tilde{B}_2$,其中$\tilde{B}_1,\tilde{B}_2$为相抵标准型则必然有$\tilde{B}_1=\tilde{B}_2$。定理\ref{xiangdi-singular}说明了这一点。

\begin{thm}
    \label{xiangdi-singular}
    同型矩阵$A,B$相抵当且仅当$r(A)=r(B)$。
\end{thm}

定理\ref{xiangdi-singular}证明如下证明\ref{pf-xiangdi-singular}。

\begin{proof}
    \label{pf-xiangdi-singular}
    充分性显然。

    必要性证明。设$A\cong B,r(A)=s,r(B)=t$,$A,B$的相抵标准型分别为$C_1,C_2$。于是有$C_1\cong C_2$,那么存在初等矩阵$P_1,\cdots,P_k,Q_1,\cdots,Q_l$使得
    $$
    P_k\cdots P_1C_1Q_1\cdots Q_l=C_2
    $$
    从而对于$P_i,i=1,\cdots,k$有$P_i'$使得$P_i' P_i=I$,故
    $$
    C_1Q_1\cdots Q_l=P_1'\cdots P_k'C_2
    $$
    令$Q=Q_1\cdots Q_l,P=P_1'\cdots P_k'$,则$P,Q$满秩且
    $$
    PC_2=\begin{bmatrix}
        P^* & 0
    \end{bmatrix}=C_1Q=\begin{bmatrix}
        Q^* \\ 0
    \end{bmatrix}
    $$
    可以推出$p_{ij}=0,i=s+1,\cdots,m;j=1,\cdots,t$且$q_{ij}=0,i=1,\cdots,s;j=t+1,\cdots,n$。此时若假设$r(A)\neq r(B)$,不失一般性,我们假定$t>s$,此时矩阵
    $$
    P=\begin{bmatrix}
        p_{11} & \cdots & p_{1t} & p_{1,t+1} & \cdots & p_{1m} \\
        \vdots & & \vdots & \vdots & & \vdots \\
        p_{s1} & \cdots & p_{st} & p_{s,t+1} & \cdots & p_{sm} \\
        0 & \cdots & 0 & p_{s+1,t+1} & \cdots & p_{s+1,m} \\
        \vdots & & \vdots & \vdots & & \vdots \\
        0 & \cdots & 0 & p_{m,t+1} & \cdots & p_{mm} \\
    \end{bmatrix}
    $$
    矩阵可以分为前$s$行和后$m-s$行,对$P$变换为阶梯型$P'$,那么前$s$行变换后得到的非零行的数量小于等于$s$,后$m-s$行经过变换得到的非零行的数量小于等于$\min\{m-s,m-t\}=m-t$行,故$P'$的非零行的数量小于等于$s+m-t$行,由于初等行变换不改变矩阵的秩,所以$r(P)=r(P')\leqslant s+m-t<t+m-t=m$,即$P$降秩,这与$P$的构造矛盾。
\end{proof}

\begin{remark}
    由此我们知道,初等变换不改变矩阵的秩,即初等行变换和初等列变换都不改变矩阵的秩。
\end{remark}

在定理\ref{pf-xiangdi-singular}的基础上,我们可以推导出定理\ref{rank-property}。

\begin{thm}
    \label{rank-property}
    对于$m\times n$的矩阵$A$,有
    \begin{enumerate}
        \item $r(A)=r(A^T)$;
        \item $r(A)=r(PA)=r(AQ)=r(PAQ)$,其中$P,Q$分别是$m,n$阶的满秩矩阵。
    \end{enumerate}
\end{thm}

	\chapter{线性方程组}

\section{向量的线性相关性}

\subsection{向量的定义}

\begin{definition}
    \label{def-vector}
    由$n$个数$a_1,\cdots,a_n$顺序构成的$n$元有序数组称为$n$元向量。记作
    \begin{equation}
        \alpha=(a_1,a_2,\cdots,a_n),\label{eq-vertor-1}
    \end{equation}
    $a_i$称为向量$\alpha$的第$a_i,i=1,2,\cdots,n$个分量。(\ref{eq-vertor-1})的形式称为行向量,除了行向量形式之外,向量还可以写成列向量的形式
    \begin{eqnarray}
        \alpha=\begin{bmatrix}
            a_1\\\vdots\\a_n
        \end{bmatrix}=(a_1,\cdots,a_n)^T.
    \end{eqnarray}
\end{definition}

向量的相等、加法、减法、数乘等操作参照矩阵的对应运算,不做赘述。

\subsection{向量的线性相关性}

\begin{definition}
    \label{def-linear-composition}
    设$\alpha_1,\cdots,\alpha_m$是$m$个$n$元向量,$k_1,\cdots,k_m$是任意$m$个数,称下列向量
    \[ \beta=k_1\alpha_1+\cdots+k_m\alpha_m \]
    是向量组的$\alpha_1,\cdots,\alpha_m$的一个线性组合。此时,也称向量$\beta$可由向量组$\alpha_1,\cdots,\alpha_m$线性表出。
\end{definition}

\begin{definition}
    \label{def-linear-relation}
    设$\alpha_1,\cdots,\alpha_m$是$m$个$n$元向量,若存在$m$个不全为零的数$k_1,\cdots,k_m$使得
    \begin{equation}
        k_1\alpha_1+\cdots+k_m\alpha_m=0\label{eq-linear-dependence}
    \end{equation}
    则称向量组$\alpha_1,\cdots,\alpha_m$线性相关,否则称为线性无关。
\end{definition}

\begin{remark}
    平行(共线)是描述两个向量之间关系的概念,而共面是描述多个向量之间关系的概念,而线性相关(线性无关)则是一个更高层次的概念。
\end{remark}

\begin{remark}
    线性相关性是从共线、共面等概念引申出来的应用于更高维向量的概念。在二维空间中,向量$\gamma,\eta$的共线即表示存在两个不全为零的实数$k_1,k_2$使得$k_1\gamma+k_2\eta=0$(这里隐含地说明了为什么零向量共线于任何向量);在三维空间中,向量$\gamma,\eta,\varepsilon$共面即表示存在三个不全为零的实数$k_1,k_2,k_3$使得$k_1\gamma+k_2\eta+k_3\varepsilon=0$。
\end{remark}

基于线性相关和线性无关的定义,可以到处一些基本结论(证明从略)。

\begin{thm}
    \label{thm-iff-vector-group-dependence}
    向量组$\alpha_1,\cdots,\alpha_m$线性相关当且仅当至少存在$\alpha_i,1\leqslant i\leqslant m$使可以由其余的向量线性表出。
\end{thm}

\begin{thm}
    \label{thm-independence-vec-grp-append}
    向量组$\alpha_1,\cdots,\alpha_m$线性无关,向量组$\beta,\alpha_1,\cdots,\alpha_m$,则$\beta$可以由$\alpha_1,\cdots,\alpha_m$线性表出,且表示法唯一。
\end{thm}

\begin{definition}
    \label{def-linear-represent}
    设$\alpha_1,\cdots,\alpha_s$与$\beta_1,\cdots,\beta_t$是两组$n$元向量,若每个$\alpha_i,i=1,2,\cdots,s$均可以由$\beta_1,\cdots,\beta_t$线性表出,则称向量组$\alpha_1,\cdots,\alpha_s$可以由向量组$\beta_1,\cdots,\beta_t$线性表出。若向量组$\alpha_1,\cdots,\alpha_s$与$\beta_1,\cdots,\beta_t$可以相互线性表出,则称$\alpha_1,\cdots,\alpha_s$与$\beta_1,\cdots,\beta_t$等价,记作
    \[ \{\alpha_1,\cdots,\alpha_s\}\cong\{\beta_1,\cdots,\beta_t\} \]
\end{definition}

向量组的等价是一种等价关系,拥有自反性、对称性和传递性。

关于向量组线性相关性,有一个重要的充分定理。

\begin{thm}
    \label{thm-important-linear-dependence}
    设$\alpha_1,\cdots,\alpha_s$是一组$n$元向量。若存在另一组$n$元向量$\beta_1,\cdots,\beta_t$有
    \begin{enumerate}
        \item $\alpha_1,\cdots,\alpha_s$可以由向量组$\beta_1,\cdots,\beta_t$线性表出;
        \item $s>t$,
    \end{enumerate}
    则向量组$\alpha_1,\cdots,\alpha_s$线性相关。
\end{thm}

定理的逆否推论如下

\begin{corollary}
    \label{col-important-linear-independence}
    设$\alpha_1,\cdots,\alpha_s$与$\beta_1,\cdots,\beta_t$是两组$n$元向量,若存在
    \begin{enumerate}
        \item $\alpha_1,\cdots,\alpha_s$可以由向量组$\beta_1,\cdots,\beta_t$线性表出;
        \item $\alpha_1,\cdots,\alpha_s$线性无关,
    \end{enumerate}
    则$s\leqslant t$。
\end{corollary}

定理\ref{thm-important-linear-dependence}的证明如下,由于定理\ref{col-important-linear-independence}是前者的逆否,只要证明了前者,则后者正确性显然。证明如下。

\begin{proof}
    \label{proof-important-linear-dependence}

    由条件1可知,存在$c_{ij},i=1,\cdots,s;j=1,\cdots,t$使得
    \[ \alpha_i=c_{i1}\beta_1+\cdots+c_{it}\beta_t \]
    设
    \[ \gamma_i=(c_{i1},\cdots,c_{it}),i=1,\cdots,s \]
    则$\gamma_1,\cdots,\gamma_s$为$s$个$t$元向量。

    由于$s>t$,根据引理\ref{lenma-vector-linear-dependence},$\gamma_1,\cdots,\gamma_s$线性相关,存在不全为零的数$k_1,\cdots,k_s$使得
    \[ k_1\gamma_1+\cdots+k_s\gamma_s=0 \]
    即
    \[ k_1c_{1j}+\cdots+k_sc_{sj}=\sum_{i=1}^sk_ic_{ij}=0,j=1,\cdots,s \]
    又因为
    \begin{equation*}
        \begin{aligned}
            \sum_{i=1}^sk_i\alpha_i
            =& \sum_{i=1}^sk_i\left(\sum_{j=1}^tc_{ij}\beta_j\right) \\
            =& \sum_{i=1}^s\sum_{j=1}^tk_ic_{ij}\beta_j \\
            =& \sum_{j=1}^t\sum_{i=1}^sk_ic_{ij}\beta_j \\
            =& \sum_{j=1}^t0\beta_j=0
        \end{aligned}
    \end{equation*}
    且$k_1,\cdots,k_t$不全为零,所以$\alpha_1,\cdots,\alpha_s$线性相关。
\end{proof}

\begin{lemma}
    \label{lenma-vector-linear-dependence}
    $m$个$n$元向量($m>n$)线性相关。
\end{lemma}

引理\ref{lenma-vector-linear-dependence}的证明就是将向量线性组合展开成为方程组,观察到方程组未知数个数大于方程个数,直接得出方程组存在非零解,从而认定,向量线性相关。

\section{向量组的秩}

\subsection{向量组的秩}

对于线性相关的向量组,秩可以描述其线性相关性的强弱程度。由此我们提出秩的概念。

\begin{definition}
    \label{def-vector-group-rank}
    设向量组$\alpha_1,\cdots,\alpha_m$有$m$个$n$元向量。其中存在$r$个向量线性无关,但是任意的$r+1$个向量都线性相关,则称向量组$\alpha_1,\cdots,\alpha_m$的值为$r$,记作$r\{\alpha_1,\cdots,\alpha_m\}=r$。
\end{definition}

向量组的秩可以来衡量向量组的线性相关性,通过其定义,容易给出如下充要条件。

\begin{thm}
    \label{thm-vector-group-dependence-rank}
    向量组$\alpha_1,\cdots,\alpha_m$线性相关当且仅当$r\{\alpha_1,\cdots,\alpha_m\}<m$。
\end{thm}

\begin{remark}
    向量组的秩为$r$并不意味着向量组中任取$r$个向量都线性无关,例如向量组
    \[
        a=(1,1,0,0),b=(1,1,0,0),c=(0,1,0,0),d=(0,1,0,0)
    \]
    其秩为$2$,但是若取部分组$a,b$,显然不是线性无关的。
\end{remark}

为了确定向量组的秩,我们引入一些相关的概念。

\begin{definition}
    \label{def-max-independence-group}
    设向量组$\alpha_1,\cdots,\alpha_m$的秩为$r$,则$\alpha_1,\cdots,\alpha_m$中任意$r$个线性无关的向量都称为其极大线性无关部分组,简称极大无关组。
\end{definition}

定义了极大无关组之后,我们可以证明向量组的所有极大无关组都是等价的。

\begin{property}
    \label{property-equvilent-mig}
    向量组与其任意一极大无关组等价。
\end{property}

\begin{proof}
    设向量组$\alpha_1,\cdots,\alpha_m$的极大无关组为$\alpha_{i_1},\cdots,\alpha_{i_r}$,显然后者可以被前者线性表出,只需证明前者可以被后者线性表出即可。

    对于向量组$\alpha_1,\cdots,\alpha_m$中的任意向量$\alpha_k$,若$\alpha_k\in\{\alpha_{i_1},\cdots,\alpha_{i_r}\}$,那么可以被其线性标出。否则,向量组$\{\alpha_k,\alpha_{i_1},\cdots,\alpha_{i_r}\}$线性相关,根据定理\ref{thm-independence-vec-grp-append},$\alpha_k$被向量组$\alpha_{i_1},\cdots,\alpha_{i_r}$唯一地线性标出。所以$\alpha_1,\cdots,\alpha_m$可以被$\alpha_{i_1},\cdots,\alpha_{i_r}$线性表出。

    故向量组与其任意一极大无关组等价。
\end{proof}

更进一步地,我们有如下定理。

\begin{thm}
    \label{thm-rank-decrease-by-linear-composition}
    若向量组$\alpha_1,\cdots,\alpha_s$可以由$\beta_1,\cdots,\beta_t$线性表出,则称向量组
    \[
        r\{\alpha_1,\cdots,\alpha_s\}\leqslant r\{\beta_1,\cdots,\beta_t\}
    \]
\end{thm}

\begin{proof}
    设向量组$\alpha_1,\cdots,\alpha_s;\beta_1,\cdots,\beta_t$的极大无关组分别为$\alpha_{i_1},\cdots,\alpha_{i_r}$和$\beta_{i_1},\cdots,\beta_{i_p}$。根据定理\ref{property-equvilent-mig},从而有
    \[
        \begin{aligned}
            \{\alpha_{i_1},\cdots,\alpha_{i_r}\}\cong&\{\alpha_1,\cdots,\alpha_s\} \\
            \{\beta_1,\cdots,\beta_t\}\cong&\{\beta_{i_1},\cdots,\beta_{i_p}\}
        \end{aligned}
    \]

    而$\beta_1,\cdots,\beta_t$又可以线性表出$\alpha_1,\cdots,\alpha_s$,所以$\beta_{i_1},\cdots,\beta_{i_p}$线性表出$\alpha_{i_1},\cdots,\alpha_{i_r}$,根据推论\ref{col-important-linear-independence}有$r\leqslant p$,从而有
    \[
        r\{\alpha_1,\cdots,\alpha_s\}\leqslant\{\beta_1,\cdots,\beta_t\}
    \]
\end{proof}

极大无关组拥有如下的等价定义。由此极大无关组的定义可以参见定义\ref{def-max-independence-group}和定义\ref{def-max-independence-group-2}。

\begin{definition}
    \label{def-max-independence-group-2}
    设$\alpha_{i_1},\cdots,\alpha_{i_r}$是向量组$\alpha_1,\cdots,\alpha_m$的一个部分组,若$\alpha_{i_1},\cdots,\alpha_{i_r}$线性无关且每个$\alpha_j(j=1,2,\cdots,m)$均可以由$\alpha_{i_1},\cdots,\alpha_{i_r}$线性表出,则$\alpha_{i_1},\cdots,\alpha_{i_r}$是向量组$\alpha_1,\cdots,\alpha_m$的极大无关组。
\end{definition}

等价性证明容易给出。只需要证明对于一个向量组,根据定义\ref{def-max-independence-group}和定义\ref{def-max-independence-group-2}定义出向量组总是相等的即可。

\begin{proof}
    设向量组$\alpha_1,\cdots,\alpha_m$,根据定义\ref{def-max-independence-group}和定义\ref{def-max-independence-group-2}定义出的极大无关组分别是$V_1,V_2$,根据定义\ref{def-max-independence-group-2},显然有$V_2\subseteq V_1$。反之,根据定义\ref{def-max-independence-group}和性质\ref{property-equvilent-mig},也能有$V_1\subseteq V_2$。所以有$V_1=V_2$,故定义\ref{def-max-independence-group}和定义\ref{def-max-independence-group-2}等价。
\end{proof}

	\appendix

\chapter{附录}

\section{符号说明}

符号说明如下

\begin{table}[!hbt]
    \centering
    \begin{tabular}{p{2cm}p{8cm}}
        \toprule
        符号 & 说明 \\
        \midrule
        $A$ & 大写字母表示矩阵。 \\
        $a,a_i,a_{ij}$ & 小写字母表示矩阵的元素,通常是标量。 \\
        \bottomrule
    \end{tabular}
\end{table}


\end{document}